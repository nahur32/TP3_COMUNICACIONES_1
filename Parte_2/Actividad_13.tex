\section{Actividad 13}

\subsection*{Una radio comercial se sintoniza en la banda de FM $97.9\ \mathrm{MHz}$. 
La radio es del tipo superheterodino, en donde la frecuencia del oscilador local ($f_{OL}$) 
está por encima de la frecuencia sintonizada y donde se emplean un amplificador de FI, 
centrado en $12.5\ \mathrm{MHz}$.}

\begin{enumerate}
  \item Determinar la frecuencia del oscilador local ($f_{OL}$).

      \[
        f_{OL} = f_s + f_{FI}
        \]
        Reemplazando por valores numéricos:
        \[
        f_{OL} = 97.9~\text{MHz} + 12.5~\text{MHz} = 110.4~\text{MHz}
        \]
        
        \[
        \boxed{f_{OL} = 110.4~\text{MHz}}
        \]
  
  \item Determinar la frecuencia imagen ($f_{im}$).

  La frecuencia imagen es aquella que, al mezclarse con el oscilador local, también produce la misma frecuencia intermedia:
        \[
        f_{im} = f_{OL} + f_{FI} = f_s + 2f_{FI}
        \]
        Reemplazando los valores:
        \[
        f_{im} = 97.9~\text{MHz} + 2(12.5~\text{MHz}) = 122.9~\text{MHz}
        \]
        
        \[
        \boxed{f_{im} = 122.9~\text{MHz}}
        \]
  
  \item Si la señal FM tiene un ancho de banda de $200\ \mathrm{kHz}$, indique los requisitos 
  mínimos para los filtros de RF y de FI, considerando que el rechazo de frecuencia imagen 
  debe alcanzar al menos $55\ \mathrm{dB}$.

    Los requisitos mínimos para los filtros son:
    
    Filtro de RF debe estar centrado en 97,9 MHz, ancho de banda de 200 kHz y rechazo $\geq$ 55 dB a 122,9 MHz; y el filtro de FI debe estar centrado en 12,5 MHz, ancho de banda de 200 kHz y alta atenuación fuera de banda para el rechazo de la frecuencia imagen.

  
  \item Si el oscilador local posee contenido apreciable de una segunda armónica, ¿cuáles son 
  las otras dos frecuencias adicionales que se reciben?

  Si el oscilador local posee contenido apreciable de una segunda armónica ($2f_{OL}$), 
esta armónica también puede mezclarse con las señales de RF.  

    Las nuevas frecuencias intermedias generadas serían:
    \[
    f_{fi1} = 2f_{OL} + f_{FI}
    \]
    \[
    f_{fi2} = 2f_{OL} - f_{FI}
    \]
    Reemplazando:
    \[
    f_{fi1} = 2(110.4) - 12.5 = 233.3~\text{MHz}
    \]
    \[
    f_{fi2} = 2(110.4) - 12.5 = 208.3~\text{MHz}
    \]
    Por lo tanto, las frecuencias adicionales coinciden con la frecuencia imagen y la frecuencia sintonizada.
    
    \[
    \boxed{
    \text{Frecuencias adicionales: } 208.3~\text{MHz y } 233.3~\text{MHz}
    }
    \]
\end{enumerate}