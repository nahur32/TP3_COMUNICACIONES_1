\section{Actividad 8}

\subsection*{Determinar cuál es la frecuencia instantánea para cada una de las siguientes señales:}

La frecuencia instantánea se define como: 
\[
 \quad f_i(t) = \frac{1}{2\pi} \frac{d\phi(t)}{dt}
\]

donde $\phi(t)$ corresponde al ángulo de la señal.


\subsection*{a) \( x_1(t) = 2\cos(20\pi t + \tfrac{\pi}{6}) \)}

\[
x_1(t) = 2\cos(20\pi t + \tfrac{\pi}{6}) \quad \Rightarrow \quad \phi_1(t) = 20\pi t + \tfrac{\pi}{6}
\]

\[
f_1(t) = \frac{1}{2\pi} \frac{d\phi_1(t)}{dt} = \frac{1}{2\pi}(20\pi) = 10~\text{Hz}
\]

\[
f_1(t) = 10~\text{Hz}
\]

\subsection*{b) \( x_2(t) = 17\cos(30\pi t + 4\cos(2\pi t)) \)}

\[
x_2(t) = 17\cos(30\pi t + 4\cos(2\pi t)) \quad \Rightarrow \quad \phi_2(t) = 30\pi t + 4\cos(2\pi t)
\]

\[
f_2(t) = \frac{1}{2\pi}\frac{d\phi_2(t)}{dt} = \frac{1}{2\pi}(30\pi - 8\pi\sin(2\pi t)) = 15 - 4\sin(2\pi t)~[\text{Hz}]
\]

\[
f_2(t) = 15 - 4\sin(2\pi t)~[\text{Hz}]
\]


\subsection*{c) \( x_3(t) = \cos(25\pi t)\cos(2\cos(7\pi t)) + \sin(25\pi t)\sin(2\cos(7\pi t)) \)}

Usando la identidad trigonométrica:
\[
\cos(x - y) = \cos x \cos y + \sin x \sin y
\]

La señal \(x_3(t)\) se puede expresar como:
\[
x_3(t) = \cos(25\pi t - 2\cos(7\pi t)) \quad \Rightarrow \quad \phi_3(t) = 25\pi t - 2\cos(7\pi t)
\]

Por lo tanto, la frecuencia instantánea de dicha señal es:
\[
f_3(t) = \frac{1}{2\pi}\frac{d\phi_3(t)}{dt} = \frac{1}{2\pi}(25\pi + 14\pi\sin(7\pi t)) = 12.5 + 7\sin(7\pi t)~[\text{Hz}]
\]

\[
f_3(t) = 12.5 + 7\sin(7\pi t)~[\text{Hz}]
\]