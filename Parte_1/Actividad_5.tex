\section{Ejercicio 5}

{\subsection*{Una versión particular de AM estéreo usa multiplexación de portadora en cuadratura. Específicamente, la portadora $A_c \cos(2\pi f_c t)$ es empleada para modular la señal suma:}

    \[
        m_1(t) = v_0 + m_l(t) + m_r(t)
    \]

\subsection*{Donde $v_0$ es un \textit{offset DC} incluido con el propósito de transmitir la componente de portadora, 
$m_l(t)$ es la señal de audio del canal izquierdo y $m_r(t)$ es la señal de audio del canal derecho.}

\subsection*{La portadora en cuadratura $A_c \sen(2\pi f_c t)$ es empleada para modular la señal diferencia:}

    \[
        m_2(t) = m_l(t) - m_r(t)
    \]

\subsection*{a) Mostrar que un detector de envolvente puede ser utilizado para recuperar la suma $m_l(t) + m_r(t)$ desde la señal multiplexada en cuadratura. 
¿Cómo puede minimizar la distorsión de la señal producida por el detector de envolvente?}


Un \textbf{detector de envolvente} extrae la envolvente de la señal, que es su amplitud instantánea (magnitud). Para una señal AM estándar, esto recupera el mensaje directamente si hay una portadora fuerte.

Para $s(t)$, la envolvente $E(t)$ es la magnitud de la componente vectorial:
\[ s(t) = I(t) \cos(2\pi f_c t) + Q(t) \sin(2\pi f_c t) \]
donde $I(t) = A_c m_1(t)$ (componente in-fase) y $Q(t) = A_c m_2(t)$ (componente en cuadratura).

La envolvente es:
\[ E(t) = \sqrt{ [A_c m_1(t)]^2 + [A_c m_2(t)]^2 } = A_c \sqrt{ m_1(t)^2 + m_2(t)^2 } \]

Sustituyendo las definiciones:
\[ E(t) = A_c \sqrt{ [v_0 + m_l(t) + m_r(t)]^2 + [m_l(t) - m_r(t)]^2 } \]

Expandimos el cuadrado:
\[ [m_l(t) + m_r(t)]^2 + 2 v_0 [m_l(t) + m_r(t)] + v_0^2 + [m_l(t) - m_r(t)]^2 = 2[m_l(t)^2 + m_r(t)^2] + 2 v_0 [m_l(t) + m_r(t)] + v_0^2 \]
donde los términos cruzados se cancelan.

Si $v_0$ es \textbf{grande} comparado con las amplitudes de $m_l(t)$ y $m_r(t)$ los términos $m_l^2 + m_r^2$ son negligibles. Entonces:
\[ E(t) \approx A_c \sqrt{ v_0^2 + 2 v_0 [m_l(t) + m_r(t)] } = A_c v_0 \sqrt{ 1 + \frac{2 [m_l(t) + m_r(t)]}{v_0} } \]

Usando la aproximación binomial $\sqrt{1 + x} \approx 1 + \frac{x}{2}$ para $x$ pequeño
donde :
\[
x = \frac{2 [m_l(t) + m_r(t)]}{v_0} \ll 1  
\]
Por lo tanto, se obtiene que:
\[ E(t) \approx A_c v_0 \left( 1 + \frac{m_l(t) + m_r(t)}{v_0} \right) = A_c v_0 + A_c [m_l(t) + m_r(t)] \]

Después de filtrar el DC (con un capacitor de acoplamiento), recuperamos $A_c [m_l(t) + m_r(t)]$, que es la suma de los canales (audio mono).

\textbf{Minimización de distorsión}: La distorsión viene de los términos no lineales (como $m_l^2 + m_r^2$) y de la componente en cuadratura $m_d(t)$. Para minimizarla:
    \begin{itemize}
        \item Haz $v_0$ lo suficientemente grande para que el índice de modulación sea bajo (e.g., < 50-70\%), asegurando que la aproximación binomial sea válida y que la envolvente no se distorsione.
        \item Limita la amplitud de $m_d(t)$ (diferencia de canales) para que sea mucho menor que $m_s(t)$.
    \end{itemize}

Esto permite compatibilidad con receptores AM mono, que ignoran la cuadratura y solo oyen la suma.



\subsection*{b) Mostrar que un detector coherente puede recuperar la diferencia $m_l(t) - m_r(t)$.}

Un \textbf{detector coherente} multiplica la señal recibida por una portadora local sincronizada y filtra pasa-bajos. Para recuperar la componente en cuadratura, usamos la portadora desfasada: $2 \sin(2\pi f_c t)$ (el factor 2 es por convención, para simplificar).

Multiplicamos $s(t)$ por $2 \sin(2\pi f_c t)$:
    \[ 
        2 s(t) \sin(2\pi f_c t) = 2 A_c m_1(t) \cos(2\pi f_c t) \sin(2\pi f_c t) + 2 A_c m_2(t) \sin^2(2\pi f_c t)
    \]

Usando identidades trigonométricas:
\begin{itemize}
    \item $2 \cos \theta \sin \theta = \sin(2\theta)$ → término de alta frecuencia (se filtra).
    \item $\sin^2(\theta) = \frac{1 - \cos(2\theta)}{2} \implies 2 \sin^2(\theta) = 1 - \cos(2\theta)$.
\end{itemize}

Se tiene que:
    \[
        Sal(t)= A_c m_1(t) \sin(4\pi f_c t) + A_c m_2(t) \left(1 + \cos(4\pi f_c t)\right)
    \]
    \[
        = A_c m_1(t) \sin(4\pi f_c t) + A_c m_2(t) - A_c m_2(t) \cos(4\pi f_c t)
    \]

Después de filtro pasa-bajos , solo queda el término banda-base de la cuadratura:
\[ 
    Sal_{PB} (t) = A_c m_d(t) \cdot 1 = A_c [m_l(t) - m_r(t)] 
\]

(Asumiendo sincronismo perfecto de fase y frecuencia entre la portadora local y la transmitida). Así, recuperamos la diferencia $m_l(t) - m_r(t)$ (multiplicada por $A_c$, que se puede ajustar con ganancia).



\subsection*{c) ¿Cómo se obtienen finalmente las señales $m_l(t)$ y $m_r(t)$ deseadas?}

Para obtener la señal $m_l (t)$, se realiza la suma entre $m_1 (t)$ y  $m_2 (t)$. Por lo tanto, se obtiene que la señal $m_l (t)$ es:
    \[
        m_1 (t) + m_2 (t) = 2m_l (t) + v_o
    \]
    
    \[
        m_l (t) = \frac{m_1 (t) + m_2 (t) + v_o}{2}
    \]

Para obtener la señal $m_r (t)$, se realiza la resta entre $m_1 (t)$ y  $m_2 (t)$. Por lo tanto, se obtiene que la señal $m_r (t)$ es:
    \[
        m_1 (t) - m_2 (t) = 2m_r (t) + v_o
    \]
    
    \[
        m_r (t) = \frac{m_1 (t) - m_2 (t) + v_o}{2}
    \]

    