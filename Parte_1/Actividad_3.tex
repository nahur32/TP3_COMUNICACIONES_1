\section{Ejercicio 3}

Con los datos del Ejercicio 2, considerar a $s(t)$ una señal de tensión y conectar a la salida de modulador un resistor de 50 $\Omega$. Cuando sea posible indicar la respuesta también en dB.

\begin{enumerate}[label=\alph*)]
    \item Calcular la potencia media de la portadora para un $\mu = 20\%$. \\
    \textbf{Rta.:} 1 W.
    \item Calcular la potencia media de la banda lateral inferior y superior para un $\mu = 20\%$. \\
    \textbf{Rta.:} 0.01 W.
    \item Calcular la potencia total de las señales moduladas para un $\mu = 20\%$. \\
    \textbf{Rta.:} 1.02 W.
    \item ¿Cuál es la relación de potencia total de banda lateral respecto la potencia total de señal modulada? \\
    \textbf{Rta.:} 1.96\%.
    \item ¿Cuál es la relación de potencia total de portadora respecto la potencia total de señal modulada? \\
    \textbf{Rta.:} 98.03\%.
    \item Repetir nuevamente los cálculos para un $\mu = 100\%$. \\
    \textbf{Rtas.:} $P_c = 1\,W$, $P_{BLI} = P_{BLS} = 0.25\,W$, $P_T = 1.5\,W$, $RP_{BL} \approx 33.3\%$, $RP_c \approx 66.6\%$.
    \item Repasar la teoría y analizar los resultados obtenidos en los puntos anteriores.
    \item Comentar las ventajas y desventajas de este tipo de modulación.
\end{enumerate}
