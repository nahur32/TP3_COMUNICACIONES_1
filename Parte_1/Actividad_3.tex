\section{Actividad 3}

\subsection*{Con los datos del Ejercicio 2, considerar a $s(t)$ una señal de tensión y conectar a la salida del modulador un resistor de $50~\Omega$. Cuando sea posible indicar la respuesta también en dB}.

\subsection*{a) Calcular la potencia media de la portadora para un $\mu = 20\%$.}

La potencia media de portadora para una resistencia de $50~\Omega$ es:
\[
    P_{\text{portadora}} = \frac{1}{2} A_c^2 = \frac{1}{2} \left(10~[V]\right)^2 \frac{1}{50~[\Omega]} = 1~[W]
\]
En dB:
\[
    P_{\text{portadora(dB)}} = 10 \log_{10}(1) = 0~[dBW]
\]

\subsection*{b) Calcular la potencia media de la banda lateral inferior y superior para un $\mu = 20\%$.}

Para una resistencia de $50~\Omega$ las potencias medias de las bandas laterales son:
\[
    P_{BL} = P_{BLS} = \frac{\mu^2 A_c^2}{8R} = \frac{(0.2)^2 (10~[V])^2}{8(50~[\Omega])} = 0.01~[W]
\]
En dB:
\[
    P_{BL(dB)} = 10 \log_{10}(0.01) = -20~[dBW]
\]

\subsection*{c) Calcular la potencia total de las señales moduladas para un $\mu = 20\%$.}

La potencia total es la suma de las tres componentes, potencia media de portadora, de banda superior y banda inferior. El resultado es:
\[
    P_{\text{total}} = P_{\text{portadora}} + P_{BLS} + P_{BLI} = 1.02~[W]
\]
En dB:
\[
    P_{\text{total(dB)}} = 10 \log_{10}(1.02) = 0.087~[dBW]
\]

\subsection*{d) ¿Cuál es la relación de potencia total de banda lateral respecto la potencia total de señal modulada?}

La relación de la potencia total de las bandas laterales con respecto a la potencia total está dada por:
\[
    RP_{\text{bandas}} = \frac{\mu^2}{2 + \mu^2} = \frac{(0.2)^2}{2 + (0.2)^2} = 0.0196 \times 100\% = 1.96\%
\]
En dB:
\[
    RP_{\text{bandas(dB)}} = 10 \log_{10}(0.0196) = -16.9~[dB]
\]

\subsection*{e) ¿Cuál es la relación de potencia total de portadora respecto la potencia total de señal modulada?}

La relación de potencia total de portadora respecto la potencia total de señal modulada se obtiene en base al resultado anterior:
\[
    RP_\textbf{portadora}=\frac{P_\text{portadora}}{P_{\text{total}}} \times 100\% = \frac{1}{1.02} \times 100\% = 98.04\%
\]
En dB:
\[
    RP_{\text{portadora(dB)}} = 10 \log_{10}\left(\frac{1}{1.02}\right) = -0.087~[dB]
\]

\subsection*{f) Repetir nuevamente los cálculos para un $\mu = 100\%$.}

Los cálculos para $\mu = 100\%$ son los siguientes:
\[
    P_{\text{portadora}} = \frac{(10~[V])^2}{2 \cdot 50~[\Omega]} = 1~[W]
\]
En dB:
\[
    P_{\text{portadora(dB)}} = 10 \log_{10}(1) = 0~[dBW]
\]

\[
    P_{BLS} = P_{BLI} = \frac{(10~[V])^2 \cdot 1^2}{8 \cdot 50~[\Omega]} = 0.25~[W]
\]
En dB:
\[
    P_{BL(dB)} = 10 \log_{10}(0.25) = -6.02~[dBW]
\]

\[
    P_{\text{total}} = P_{\text{portadora}} + P_{BLS} + P_{BLI} = 1.5~[W]
\]
En dB:
\[
    P_{\text{total(dB)}} = 10 \log_{10}(1.5) = 1.76~[dBW]
\]

\[
    RP_{\text{bandas}} = \frac{\mu^2}{2 + \mu^2} = \frac{1^2}{2 + 1^2} = \frac{1}{3} \times 100\% = 33.33\%
\]
En dB:
\[
    RP_{\text{bandas(dB)}} = 10 \log_{10}\left(\frac{1}{3}\right) = -4.77~[dB]
\]

\[
    RP_\textbf{portadora}=\frac{P_\text{portadora}}{P_{\text{total}}} \times 100\% = \frac{1}{1.5} \times 100\% = 66.66\%
\]
En dB:
\[
    RP_{\text{portadora(dB)}} = 10 \log_{10}\left(\frac{1}{1.5}\right) = -1.76~[dB]
\]


\subsection*{g) Repasar la teoría y analizar los resultados obtenidos en los puntos anteriores.}

    Al analizar los resultados obtenidos se observa que, para valores bajos del índice de modulación $\mu$, gran parte de la potencia total se concentra en la portadora. Esto no es deseable, ya que aproximadamente el $98{,}04\%$ de la potencia total se gasta en la portadora, la cual no transporta información. 
A medida que el valor de $\mu$ aumenta, la potencia de la portadora permanece constante, pero la proporción de potencia correspondiente a las bandas laterales se incrementa. En consecuencia, una mayor parte de la potencia total se destina a las bandas laterales, que son las que efectivamente contienen la información del mensaje. 

Por lo tanto, se busca trabajar con valores de $\mu$ lo más altos posible sin superar la unidad, para evitar la sobremodulación y la distorsión de la envolvente.

    
\subsection*{h) Comentar las ventajas y desventajas de este tipo de modulación.}

Las ventajas y desventajas de este tipo de modulación, son las siguientes:

\begin{itemize}
    \item Ventajas: La simplicidad que supone a la hora de implementar el receptor.
    \item Desventajas:
        \begin{itemize}
            \item La transmisión de la onda portadora supone una pérdida de potencia para la señal resultante.
            \item La modulación de amplitud malgasta ancho de banda ya que requiere un ancho de banda de transmisión igual al doble del ancho de banda del mensaje.
        \end{itemize}
\end{itemize}