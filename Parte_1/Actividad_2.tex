\section{Ejercicio 2}

Un transmisor con modulación de amplitud tiene como entrada una señal

\[
m(t) = A \cos(2 \pi f t) \quad \text{con } A = 3\,V \text{ y } f = 600\,Hz.
\]

Si la portadora a modular posee una amplitud de 10 V y una frecuencia de 1 kHz:

\begin{enumerate}[label=\alph*)]
    \item Calcular de manera analítica la señal de salida del modulador en tiempo y en frecuencia, graficando además los resultados. Suponer un 90\% de porcentaje de modulación. ¿Es posible realizar una detección de envolvente sobre la señal $s(t)$ o se requiere una detección coherente?
    \item Modificar la frecuencia de portadora a $f = 10 f_m$ y graficar $s(t)$ para el porcentaje de modulación 90\%, 60\%, 20\% y 110\%. ¿Es posible realizar una detección de envolvente sobre la señal $s(t)$ para todos los casos propuestos? Justificar en el caso que no sea factible la detección.
\end{enumerate}
